\documentclass[12pt,t]{beamer}
\usepackage[utf8]{inputenc}
\usepackage[catalan]{babel}
\usepackage{verbatim}
\usepackage{hyperref}
\usepackage{amsfonts,amssymb,amsmath,amsthm, wasysym}
\usepackage{listings}
\usepackage[T1]{fontenc}        
\usepackage{pgf}
\usepackage{epsdice}
\usepackage{pgfpages}
\usepackage{tikz}
\setbeamertemplate{caption}[numbered]
\setbeamertemplate{navigation symbols}{}


\newcommand{\red}[1]{\textcolor{red}{#1}}
\newcommand{\green}[1]{\textcolor{green}{#1}}
\newcommand{\blue}[1]{\textcolor{blue}{#1}}
\newcommand{\gray}[1]{\textcolor{gray}{#1}}
\renewcommand{\emph}[1]{{\color{red}#1}}

\setbeamertemplate{frametitle}
{\begin{centering}
\smallskip
\color{blue}
\textbf{\insertframetitle}
\end{centering}
}
\usecolortheme{rose}
\usecolortheme{dolphin}
\mode<presentation>


\newcommand{\CC}{\mathbb{C}}
\newcommand{\RR}{\mathbb{R}}
\newcommand{\ZZ}{\mathbb{Z}}
\newcommand{\NN}{\mathbb{N}}
\newcommand{\KK}{\mathbb{K}}
\newcommand{\MM}{\mathcal{M}}
%\newcommand{\dbinom}{\displaystyle\binom}

\newcommand{\limn}{{\displaystyle \lim_{n\to\infty}}}
\renewcommand{\leq}{\leqslant}
\renewcommand{\geq}{\geqslant}
\def\tendeix{{\displaystyle\mathop{\longrightarrow}_{\scriptscriptstyle
n\to\infty}}}

\newcommand{\matriu}[1]{\left(\begin{matrix} #1 \end{matrix}\right)}

% \newcommand{\qed}{\hbox{}\nobreak\hfill\vrule width 1.4mm height 1.4mm depth 0mm
%     \par \goodbreak \smallskip}
%
% %
\theoremstyle{plain}
\newtheorem{teorema}{Teorema}
\newtheorem{prop}{Proposició}
\newtheorem{cor}{Coro\l.lari}
\theoremstyle{definition}
\newtheorem{exemple}{Exemple}
\newtheorem{defin}{Definició}
\newtheorem{obs}{Observació}

\newcounter{seccions}
\newcommand{\seccio}[1]{\addtocounter{seccions}{1}
\medskip\par\noindent\emph{\theseccions.
#1}\smallskip\par }

\newcommand{\EM}{\Omega}
\newcommand{\PP}{\mathcal{P}}

\usepackage{listings}

\lstset{
language=R, %
aboveskip=1 ex, %
belowskip=-4ex, %
linewidth=\textwidth,
%xleftmargin=0.5cm,
%xrightmargin=0.5cm,
basicstyle={\footnotesize\ttfamily}, %
commentstyle=\ttfamily\color{red}, %
numbers=none, %
%morecomment=[l]{E},
numberstyle=\ttfamily\color{gray}\footnotesize,
stepnumber=1,
numbersep=5pt,
backgroundcolor=\color{gray!15},%15
showspaces=false,
%showstringspaces=false,
showtabs=false,
frame=single, %????
framerule=0pt,
keepspaces=true,
tabsize=2,
captionpos=b, %
breaklines=true, %
breakatwhitespace=false,  %
showstringspaces=false,
title=\lstname,
literate={{á}{{\'a}}1 {é}{{\'e}}1 {í}{{\'i}}1 {ó}{{\'o}}1 {ú}{{\'u}}1
  {Á}{{\'A}}1 {É}{{\'E}}1 {Í}{{\'I}}1 {Ó}{{\'O}}1 {Ú}{{\'U}}1
  {à}{{\`a}}1 {è}{{\`e}}1 {ì}{{\`i}}1 {ò}{{\`o}}1 {ù}{{\`u}}1
  {À}{{\`A}}1 {È}{{\'E}}1 {Ì}{{\`I}}1 {Ò}{{\`O}}1 {Ù}{{\`U}}1
  {ä}{{\"a}}1 {ë}{{\"e}}1 {ï}{{\"i}}1 {ö}{{\"o}}1 {ü}{{\"u}}1
  {Ä}{{\"A}}1 {Ë}{{\"E}}1 {Ï}{{\"I}}1 {Ö}{{\"O}}1 {Ü}{{\"U}}1 {ñ}{{\~n}}1 {Ñ}{{\~N}}1
  {â}{{\^a}}1 {ê}{{\^e}}1 {î}{{\^i}}1 {ô}{{\^o}}1 {û}{{\^u}}1
  {Â}{{\^A}}1 {Ê}{{\^E}}1 {Î}{{\^I}}1 {Ô}{{\^O}}1 {Û}{{\^U}}1
  {œ}{{\oe}}1 {Œ}{{\OE}}1 {æ}{{\ae}}1 {Æ}{{\AE}}1 {ß}{{\ss}}1
  {ç}{{\c c}}1 {Ç}{{\c C}}1 {ø}{{\o}}1 {å}{{\r a}}1 {Å}{{\r A}}1
  {€}{{\EUR}}1 {£}{{\pounds}}1 {¡}{{\textexclamdown}}1 {¿}{{\textquestiondown}}1 {!}{{!}}1
  {·}{{\textperiodcentered}}1}
 }
 
\lstdefinestyle{warning}{basicstyle={\footnotesize\ttfamily\color{red}}}



\title[\red{Matemàtiques II}]{}
\author[]{}
\date{}



\begin{document}
\beamertemplatedotitem


\begin{frame}
\vfill
\begin{center}
\gray{\LARGE\bf ANOVA de blocs}
\end{center}
\end{frame}


\begin{frame}
\frametitle{Exemple}

En un estudi es volgué determinar quina varietat de mongeta (d'entre 6 varietats) era la més adient per  cultivar en una granja, atès el seu tipus de terra. \medskip

Per evitar que els llocs on se sembrassin les diferents mongeteres poguessin afectar els resultats, en prengueren 4 quadrats (\red{blocs}) de terreny de la granja, cada un es quadriculà en 6 parts i a cada bloc se li assignà de manera aleatòria  a cada quadrat un tipus diferent de mongetera.

\end{frame}


\begin{frame}
\frametitle{Exemple}

Les produccions per quadrat (en kg)
\begin{center}
\begin{tabular}{r|cccccc}
\multicolumn{1}{c}{} &  \multicolumn{6}{c}{Varietat de mongeta}\\
Bloc & 1 & 2 & 3 & 4 & 5 & 6\\ \hline
1 &  15.9  &  15.5  &  13.0  &  10.3  &  10.0  &   6.4\\
2 &   16.9  &  18.1  &  12.3  &  13.6   & 10.1  &  10.0\\
3 &   15.0  &  16.8   &  9.0   & 11.2   &  8.2    & 6.5\\
4 &   13.9  &  13.0  &  11.3  &  11.4 &    8.6   &  8.7\\
\end{tabular}
\end{center}\medskip

\blue{La producció mitjana de les diferents varietats de mongeta en aquest tipus de terra és la mateixa?}
\end{frame}


\begin{frame}
\frametitle{Exemple}

\red{Variable poblacional global}: 
\begin{itemize}
\item $X$: Prenc un quadrat sembrat de mongeteres (de les característiques dels quadrats emprats en aquest experiment) i mesur la seva producció (en kg de mongetes)\medskip
\end{itemize}

\red{Subpoblacions}: 
\begin{itemize}
\item Les 6 varietats de mongeta (1,\ldots,6)\medskip
\end{itemize}

\red{Variables d'interès}: 
\begin{itemize}
\item $X_i$: Prenc un quadrat sembrat amb mongeteres de varietat $i$ i mesur la seva producció ($i=1,\ldots,6$)
\end{itemize}

\red{Contrast}: 
$$
\left\{
\begin{array}{l}
H_0 : \mu_1=\mu_{2}=\cdots=\mu_{6} \\
H_1 : \mbox{Hi ha  }i,j\mbox{ tals que }  \mu_i \not=\mu_j
\end{array}
\right.
$$
\end{frame}




\begin{frame}
\frametitle{ANOVA de blocs}

En un experiment amb \emph{disseny d'ANOVA de blocs}:

\begin{itemize}
\item Empram els nivells (\red{tractaments}) d'\red{un únic factor} per classificar la població en $k\geq 3$ subpoblacions
\begin{itemize}
\item \blue{El tipus de mongetera, amb $k=6$ nivells}
\end{itemize}


\item Prenem una mostra aleatòria de $b$ \emph{blocs}: conjunts de $k$ subjectes aparellats
\begin{itemize}
\item \blue{Els $b=4$ terrenys que quadriculam  en $k=6$ quadrats}
\end{itemize}


\item Dins cada bloc, assignam aleatòriament a cada subjecte un tractament, de manera que cada tractament s'empri exactament un cop dins  cada bloc (\red{complet aleatori})
\begin{itemize}
\item \blue{Dins cada bloc, assignam aleatòriament a cada quadrat un tractament diferent}
\end{itemize}
\end{itemize}

L'\red{ANOVA de blocs} generalitza el contrast de 2 mitjanes  amb mostres aparellades a $k$ mitjanes  amb mostres aparellades
\end{frame}



\begin{frame}
\frametitle{Situació general}

Mesuram una variable $X$ sobre $k$ subpoblacions (\red{tractaments})\smallskip

\begin{itemize}
\item \red{$\mu$}: mitjana poblacional  de $X$ en tota la població
\smallskip

\item \red{$X_i$}: Prenc un individu del nivell $i$-èsim i hi mesur la $X$, $i=1,\ldots,k$; la seva mitjana és 
\red{$\mu_i$}\smallskip

\item \red{$X_{\bullet j}$}: Prenc un individu del bloc $j$-èsim i hi mesur la $X$, $j=1,\ldots,b$; la seva mitjana és  \red{$\mu_{\bullet j}$}
\smallskip

\item \red{$X_{ij}$}: Prenc l'individu del nivell $i$-èsim del bloc $j$-èssim i hi mesur la $X$; la seva mitjana és 
 \red{$\mu_{ij}$}
\end{itemize}\pause

\red{Contrast}:
$$
\left\{
\begin{array}{l}
H_0 : \mu_1=\mu_{2}=\cdots=\mu_{k}\ \only<3>{\red{ =\mu}} \\
H_1 : \mbox{Hi ha  }i,j\mbox{ tals que }  \mu_i \not=\mu_j
\end{array}
\right.
$$
\end{frame}


\begin{frame}
\frametitle{Situació general}

Les dades es presenten en una taula:
\begin{center}
\begin{tabular}{c|cccc}
\multicolumn{5}{c}{\hphantom{Blocs} Tractaments}
\\  Bloc & Tract.  1 &Tract.
 2 & \ldots &Tract.  k \\\hline
 1 &$X_{11}$&$X_{21}$&$\ldots$&$X_{k1}$\\
2&$X_{12}$&$X_{22}$&$\ldots$&$X_{k2}$\\
$\vdots$&$\vdots$&$\vdots$&$\vdots$&$\vdots$\\
$b$&$X_{1b}$&$X_{2b}$&$\ldots$&$X_{kb}$\\\hline
\end{tabular}
\end{center}
\medskip

\red{$X_{ij}$}: 
\begin{itemize}
\item La $i$ representa la columna: el tractament
\item La $j$ representa la filera: el bloc
\end{itemize}
\end{frame}

\begin{frame}
\frametitle{Situació general}

\begin{itemize}

\item $\red{\overline{X}_{i}}$: mitjana mostral del tractament $i$-èsim 
$$
\overline{X}_{i}=\dfrac{\sum_{j=1}^b X_{ij}}{b}
$$

\item $\red{\overline{X}_{\bullet j}}$: mitjana mostral del bloc $j$-èsim 
$$
\overline{X}_{\bullet j}=\dfrac{\sum_{i=1}^k X_{ij}}{k}
$$

\item $\red{\overline{X}}$: mitjana mostral global 
$$
\overline{X}=\dfrac{\sum_{i=1}^k\sum_{j=1}^b X_{ij}}{k\cdot b}
$$
\end{itemize}
\end{frame}

\begin{frame}
\frametitle{Situació general}

\begin{center}
\begin{tabular}{c|cccc|c}
\multicolumn{6}{c}{Tractaments}
\\  Bloc & Tract.  1 &Tract.
 2 & \ldots &Tract.  k & $\overline{X}_{\bullet j}$ \\\hline
 1 &$X_{11}$&$X_{21}$&$\ldots$&$X_{k1}$& $\vphantom{\int^A}\overline{X}_{\bullet 1}$\\
2&$X_{12}$&$X_{22}$&$\ldots$&$X_{k2}$ & $\overline{X}_{\bullet 2}$\\
$\vdots$&$\vdots$&$\vdots$&$\vdots$&$\vdots$&$\vdots$\\
$b$&$X_{1b}$&$X_{2b}$&$\ldots$&$X_{kb}$ & $\overline{X}_{\bullet b}$\\\hline
$\vphantom{\int^A}\overline{X}_i$ & $\overline{X}_1$ & $\overline{X}_2$ &  $\ldots$ & $\overline{X}_k$\\[-2ex]
\multicolumn{6}{c}{$\underbrace{\hphantom{\overline{X}_{11}+\overline{X}_{11}+\overline{X}_{11}+\overline{X}_{11}+\overline{X}_{11}}}_{\displaystyle \overline{X}}$}
\end{tabular}
\end{center}

\end{frame}



\begin{frame}
\frametitle{Exemple}
Emmagatzemam les dades en  un \textsl{dataframe} amb tres variables:
\begin{itemize}
\item \texttt{Prod}: la producció
\item \texttt{Mong}: la varietat de mongetera (un \red{factor})
\item \texttt{Bloc}: el bloc (un \red{factor})
\end{itemize}


\begin{center}
\begin{tabular}{r|cccccc}
\multicolumn{1}{c}{} &  \multicolumn{6}{c}{Var. mongetes}\\
Bloc & 1 & 2 & 3 & 4 & 5 & 6\\ \hline
1 &  15.9  &  15.5  &  13.0  &  10.3  &  10.0  &   6.4\\
2 &   16.9  &  18.1  &  12.3  &  13.6   & 10.1  &  10.0\\
3 &   15.0  &  16.8   &  9.0   & 11.2   &  8.2    & 6.5\\
4 &   13.9  &  13.0  &  11.3  &  11.4 &    8.6   &  8.7\\
\end{tabular}
\end{center}
\end{frame}


\begin{frame}[fragile]
\frametitle{Exemple}


\begin{lstlisting}
> Prod=c(15.9,15.5,13.0,10.3,10.0,6.4,
     16.9,18.1,12.3,13.6,10.1,10.0,
     15.0,16.8,9.0,11.2,8.2,6.5,
     13.9,13.0,11.3,11.4,8.6,8.7)
> Mong=as.factor(rep(1:6,times=4))
> Mong
 [1] 1 2 3 4 5 6 1 2 3 4 5 6 1 2 3 4 5 6 1 2 3 4 5 6
Levels: 1 2 3 4 5 6
> Bloc=as.factor(rep(1:4,each=6))
> Bloc
 [1] 1 1 1 1 1 1 2 2 2 2 2 2 3 3 3 3 3 3 4 4 4 4 4 4
Levels: 1 2 3 4
\end{lstlisting}
\end{frame}


\begin{frame}[fragile]
\frametitle{Exemple}


\begin{lstlisting}
> Dades=data.frame(Prod,Mong,Bloc)
> str(Dades)
'data.frame':	24 obs. of  3 variables:
 $ Prod: num  15.9 15.5 13 10.3 10 6.4 16.9 18.1 12.3 13.6 ...
 $ Mong: Factor w/ 6 levels "1","2","3","4",..: 1 2 3 4 5 6 1 2 3 4 ...
 $ Bloc: Factor w/ 4 levels "1","2","3","4": 1 1 1 1 1 1 2 2 2 2 ...
\end{lstlisting}
\end{frame}


\begin{frame}[fragile]
\frametitle{Exemple}\vspace*{-2ex}


\begin{lstlisting}
> Xb.i=aggregate(Prod~Mong,data=Dades,mean)
> Xb.i
  Mong   Prod
1    1 15.425
2    2 15.850
3    3 11.400
4    4 11.625
5    5  9.225
6    6  7.900
> Xb.bj=aggregate(Prod~Bloc,data=Dades,mean)
> Xb.bj
  Bloc     Prod
1    1 11.85000
2    2 13.50000
3    3 11.11667
4    4 11.15000
> Xb=mean(Prod)
> Xb
[1] 11.90417
\end{lstlisting}
\end{frame}

\begin{frame}
\frametitle{Exemple}

\begin{itemize}
\item Mitjanes mostrals dels tractaments:\smallskip

\begin{center}
\begin{tabular}{cccccc}
${\overline{X}_{1}}$ & ${\overline{X}_{2}}$ & ${\overline{X}_{3}}$ & ${\overline{X}_{4}}$ & ${\overline{X}_{5}}$ & ${\overline{X}_{6}}$  \\\hline
15.425 & 15.85 & 11.4 & 11.625&  9.225&  7.9  
\end{tabular}
\end{center}
\smallskip


\item Mitjanes mostrals  dels blocs:

\begin{center}
\begin{tabular}{cccc}
${\overline{X}_{\bullet1}}$ & ${\overline{X}_{\bullet2}}$ & ${\overline{X}_{\bullet3}}$ & ${\overline{X}_{\bullet4}}$   \\\hline
11.85 & 13.5 & 11.12 & 11.15
\end{tabular}
\end{center}
\smallskip


\item Mitjana mostral global: $\overline{X}=11.9$

\end{itemize}
\end{frame}




\begin{frame}
\frametitle{Condicions necessàries}

Per poder fer una ANOVA de blocs, cal que:\medskip

\begin{itemize}

\item Les $k\cdot b$ observacions constitueixen mostres aleatòries, cadascuna de mida $1$, de les $k\cdot b$ variables $X_{ij}$
\smallskip

\item Les variables $X_{ij}$ són totes normals amb la mateixa variància $\sigma^2$
\smallskip

\item L'efecte dels blocs i els tractaments és \emph{additiu}: no hi ha \emph{interacció} entre els blocs i els tractaments:\vspace*{1ex}

\begin{quote} 
\blue{Per a cada parell de nivells $i_1,i_2$ i per a cada parell de blocs $j_1,j_2$ \vspace*{-1ex}
$$
\mu_{i_1j_1}-\mu_{i_2j_1}=\mu_{i_1j_2}-\mu_{i_2j_2}
$$}
\end{quote} 
\end{itemize}\vspace*{-5ex}

Cap d'elles no es pot contrastar, per tant l'experimentador decideix si se satisfan o no segons la seva experiència
\end{frame}


\begin{frame}
\frametitle{Interacció?}


Mesuram l'efecte d'un analgèsic A (en disminució del grau de dolor) en homes i dones 
\begin{itemize}
\item $\mu_B$: grau mitjà de dolor abans de prendre A 

\item $\mu_{BD}$: grau mitjà de dolor de les dones abans de prendre A 


\item $\mu_{BH}$: grau mitjà de dolor dels homes abans de prendre A 

\item $\mu_A$: grau mitjà de dolor després  de prendre A 

\item $\mu_{AD}$: grau mitjà de dolor de les dones després de prendre A 


\item $\mu_{AH}$: grau mitjà de dolor dels homes després  de prendre A
\end{itemize}\pause
\begin{itemize}
\item \emph{No interacció:} $\mu_{BD}-\mu_{AD}=\mu_{BH}-\mu_{AH}\pause =\mu_{B}-\mu_A$\pause\smallskip

\item \emph{Sí interacció:} $\mu_{BD}-\mu_{AD}\neq \mu_{BH}-\mu_{AH}$
\end{itemize}

\end{frame}

\begin{frame}
\frametitle{Interacció?}


\blue{Exercici}: En un contrast obteniu un p-valor 0.0001. Amb $\alpha=0.05$:

\begin{enumerate}
\item \textbf{(1 punt)} Acceptau o rebutjau $H_0$?  \only<2,3,4>{\red{Rebutjau}}
\item \textbf{(1 punt)} Quin tipus d'error pot ser que cometeu? \only<2,3,4>{\red{Tipus I}}
\end{enumerate}\medskip

\only<3>{\red{No interacció}:
\begin{itemize}
\item Si teniu bé les dues, 2 punts
\item Si en teniu una bé i una malament, 1 punt 
\item Si teniu malament les dues, 0 punts
\end{itemize}}

\only<4>{\red{Interacció}
\begin{itemize}
\item Si teniu bé les dues, 2 punts
\item Si teniu bé la 1a i malament la 2a, 1 punt
\item Si en teniu bé la 2a i malament la 1a,  0 punts (respostes inconsistents)
\item Si teniu malament les dues, 1 punt (respostes consistents)
\end{itemize}}
\end{frame}




\begin{frame}
\frametitle{{Model}}\vspace*{-5ex}

$$
\hspace*{-1ex} X_{ij}- \mu= (\mu_{i}-\mu) +(\mu_{\bullet j}-\mu) + E_{ij}, \ \mbox{\footnotesize $i=1,\ldots,k,\ j=1,\ldots,b$}
$$
on:
\begin{itemize}
\item  \red{$\mu_{i}-\mu$}: \red{Efecte del tractament $i$-èsim}
\medskip

\item  \red{$\mu_{\bullet j}-\mu$}:  \red{Efecte del bloc $j$-èsim}
\medskip

\item \red{$E_{ij}$} ($=X_{ij}-\mu_{i}-\mu_{\bullet j}+\mu$): \red{Residu}, \red{Error aleatori}
\end{itemize}
\end{frame}


\begin{frame}
\frametitle{Identitat de les sumes de quadrats}\vspace*{-3ex} 

\begin{teorema}
$SS_{Total}=SS_{Tr}+SS_{Blocs}+SS_E$
\end{teorema}
\vspace*{-2ex}

\begin{itemize}
\item $\red{SS_{Total}} = \sum\limits_{i=1}^k\sum\limits_{j=1}^b (X_{ij}-
\overline{X})^2$, és la \red{Suma de Total de Quadrats}: \blue{variabilitat global de la mostra} 

\item $\red{SS_{Tr}}=b\sum\limits_{i=1}^k
(\overline{X}_{i}-\overline{X})^2$, és la \red{Suma de Quadrats dels Tractaments}: \blue{variabilitat 
de les mitjanes dels tractaments} 

\item $\red{SS_{Blocs}}=k\sum\limits_{j=1}^b (\overline{X}_{\bullet
j}-\overline{X})^2$, és la \red{Suma de Quadrats dels Blocs}: \blue{variabilitat 
de les mitjanes dels blocs} 

\item $\red{SS_E}= \sum\limits_{i=1}^k\sum\limits_{j=1}^b (X_{ij} - \overline{X}_{i}-
\overline{X}_{\bullet j}+\overline{X})^2$, és la \red{Suma de Quadrats dels Residus} o \red{dels Errors}: \blue{variabilitat  deguda a factors aleatoris}
\end{itemize}

\end{frame}


\begin{frame}[fragile]
\frametitle{Exemple}\vspace*{-1ex} 


\begin{lstlisting}
> k=6; b=4
> SS.Tot=sum((Prod-Xb)^2)
> SS.Tot
[1] 252.2496
> SS.Tr=b*sum((Xb.i[,2]-Xb)^2)
> SS.Tr
[1] 206.0371
> SS.Bl=k*sum((Xb.bj[,2]-Xb)^2)
> SS.Bl
[1] 22.43125
> SSE=sum((Prod-Xb.i[,2]-rep(Xb.bj[,2],each=k)+Xb)^2)
> SSE
[1] 23.78125
\end{lstlisting}\medskip

Identitat de les sumes de quadrats?


\begin{lstlisting}
> SS.Tr+SS.Bl+SSE
[1] 252.2496
\end{lstlisting}

\end{frame}




\begin{frame}
\frametitle{Contrast}\vspace*{-1ex} 

\begin{itemize}
\item \red{Quadrat mitjà dels tractaments}: 
$$
\red{MS_{Tr}}=\dfrac{SS_{Tr}}{k-1}
$$


\item \red{Quadrat mitjà dels errors}: 
$$
\red{MS_E} = \dfrac{SS_E}{(b-1) (k-1)}
$$


\item (A més, R calcula) \red{Quadrat mitjà dels blocs}:
$$
\red{MS_{Blocs}}=\dfrac{SS_{Blocs}}{b-1}
$$
\end{itemize}
\end{frame}



\begin{frame}
\frametitle{Contrast}

Si se satisfan les condicions necessàries per fer una ANOVA de blocs:
$$
\begin{array}{l}
\displaystyle E(MS_{Tr})=\sigma^2 + \dfrac{b}{k-1}\sum_{i=1}^k (\mu_{i}-\mu)^2 \\
E(MS_E)=\sigma^2
\end{array}
$$


En particular, \red{$MS_E$ estima la variància comuna $\sigma^2$}
\medskip

Si $H_0:\mu_{1}=\cdots =\mu_{k}(=\mu)$ és certa,
$$
\dfrac{b}{k-1}\sum\limits_{i=1}^k (\mu_{i}-\mu)^2 = 0,
$$
i si $H_0$ no és certa, aquesta quantitat és $>0$
\end{frame}


\begin{frame}
\frametitle{Contrast}

Prenem com a \emph{estadístic de contrast} 
$$
\red{F=\frac{MS_{Tr}}{MS_E}}
$$

Si $H_0$ és certa:
\medskip

\begin{itemize}
\item la seva distribució és $F_{k-1,(b-1)(k-1)}$ (F de Fisher-Snedecor 
amb $k-1$ i $(b-1)(k-1)$ graus de llibertat)
\medskip

\item el seu valor serà proper a $1$
\end{itemize}
\medskip

A més, si $k=2$, $F$ és igual al quadrat de l'estadístic del test t de 2 mostres aparellades\medskip

\red{Rebutjarem la hipòtesi nu\l.la si $F$ és molt gran}:
$$
\text{p-valor}=P(F_{k-1,(b-1)(k-1)}\geq F)
$$



\end{frame}


\begin{frame}
\frametitle{Contrast}

\begin{enumerate}
\item Calculam 
$$
SS_{Tr},\ SS_E
$$

\item Calculam 
$$
\hspace*{-5ex} MS_{Tr}=\frac{SS_{Tr}}{k-1},\
MS_E=\frac{SS_E}{(b-1)(k-1)}
$$

\item Calculam 
$$
F=\frac{MS_{Tr}}{MS_E}
$$

\item Calculam el p-valor
$$
P(F_{k-1,(b-1)(k-1)}\geq F)
$$

\item Si el p-valor és més petit que el nivell de significació $\alpha$,  rebutjam $H_0$ i concloem que no totes les mitjanes són iguals. En cas contrari, acceptam $H_0$.
\end{enumerate}
\end{frame}





\begin{frame}
\frametitle{Exemple}

\begin{center}
\begin{tabular}{cccccc}
k & b & $SS_{Total}$ & $SS_{Tr}$ & $SS_{Blocs}$ & $SS_E$\\ \hline
6 & 4 & 252.24  & 206.04 & 22.43  & 23.78
\end{tabular}
\end{center}

\begin{itemize}
\item $MS_{Tr}=\dfrac{SS_{Tr}}{k-1}=\dfrac{206.04}{5}=41.21    $
\medskip

\item $MS_E = \dfrac{SS_E}{(b-1) (k-1)}=\dfrac{23.78}{3\cdot 5}=1.59$
\medskip

\item $MS_{Blocs}=\dfrac{SS_{Blocs}}{b-1}=\dfrac{23.78}{3}=7.48$
\medskip

\item $F=\dfrac{MS_{Tr}}{MS_E}=26$
\medskip

\item p-valor: $P(F_{5,15}\geq 26)=\texttt{1-pf(26,5,15)}=7\cdot 10^{-7}$
\medskip

\item Conclusió: Hem obtingut evidència estadística que les produccions mitjanes per als diferents tipus de mongetera no són totes iguals (ANOVA de blocs, p-valor)

\end{itemize}
\end{frame}

\begin{frame}
\frametitle{Taula ANOVA}

Una ANOVA de blocs es resumeix en una \red{taula ANOVA}:
\begin{center}
\small \begin{tabular}{llllll}
\hline
Origen&Graus de&Suma de&Quadrats&Estadístic & p-valor\\
variació&llibertat&quadrats&mitjans& & \\\hline
Tracts.&$k-1$ & $SS_{Tr}$&$MS_{Tr}$&$F$ & p-valor\\[2ex]
Blocs &$b-1$&$SS_{Blocs}$&$MS_{Blocs}$& &\\[2ex]
Errors&$(b-1)(k-1)$&$SS_{E}$&$MS_E$& &\\
\hline
\end{tabular}
\end{center}
\pause\medskip

\blue{Exemple}:

\begin{center}
\small 
\small \begin{tabular}{llllll}
\hline
Origen&Graus de&Suma de&Quadrats&Estadístic & p-valor\\
variació&llibertat&quadrats&mitjans& & \\\hline
Tracts.& 5 & $206.04$&$41.21$& 25.99 &  $7\cdot 10^{-7}$\\[2ex]
Blocs& 3 &$22.43$&$7.48$& &\\[2ex]
Errors& 15 &$23.78$&$1.59$& &\\
\hline
\end{tabular}
\end{center}

\end{frame}



\begin{frame}[fragile]
\frametitle{Amb R}

S'aplica \red{\texttt{summary(aov(\ ))}} a la fórmula que separa la variable numèrica per la \blue{suma (+) dels tractaments i els blocs}\medskip

 
\begin{lstlisting}
> summary(aov(Prod~Mong+Bloc, data=Dades))
          Df Sum Sq Mean Sq F value   Pr(>F)    
Mong       5 206.04   41.21  25.992 6.89e-07 ***
Bloc       3  22.43    7.48   4.716   0.0164 *  
Residuals 15  23.78    1.59                     
---
Signif. codes:  0 `***' 0.001 `**' 0.01 `*' 0.05 `.' 0.1 ` ' 1
\end{lstlisting}

El p-valor de la filera \texttt{Bloc} contrasta si hi ha diferències entre les mitjanes dels blocs.
\end{frame}







\begin{frame}
\frametitle{Comparacions posteriors per parelles}

Si rebutjam $H_0$, podem demanar-nos quins tractaments donen mitjanes diferents
\medskip

Podem emprar un test t, fent cada comparació \emph{per a mostres aparellades} i emprant un ajust del p-valor (Bonferroni, Holm, \ldots)\medskip

Amb R es fa amb \texttt{pairwise.t.test} indicant-hi que \texttt{paired=TRUE}

\end{frame}


\begin{frame}[fragile]
\frametitle{Exemple}\vspace*{-1ex}

\begin{lstlisting}
> pairwise.t.test(Dades$Prod, Dades$Mong,
  paired=TRUE, p.adjust.method="bonferroni")

	Pairwise comparisons using paired t tests 

data:  Dades$Prod and Dades$Mong 

  1      2      3      4      5     
2 1.0000 -      -      -      -     
3 0.2208 0.7870 -      -      -     
4 0.1542 0.2775 1.0000 -      -     
5 0.0068 0.1052 0.3139 0.6583 -     
6 0.0614 0.1310 0.6459 0.0432 1.0000

P value adjustment method: bonferroni 
\end{lstlisting}

Només trobam evidència que $\mu_1\neq \mu_5$ i $\mu_4\neq \mu_6$
\end{frame}

\begin{frame}[fragile]
\frametitle{Contrast no paramètric}

Si no podem  aplicar ANOVA de blocs perquè sospitem que no se satisfan les
condicions necessàries, \red{cal emprar un test no paramètric}\medskip

El més popular és el \red{test de Friedman} (generalitza el test de Wilcoxon amb mostres aparellades a més de 2 mostres), implementat a la funció \red{\texttt{friedman.test}} (\blue{cal substituir $+$ per $|$ a la fórmula})\medskip

\begin{lstlisting}
> friedman.test(Prod~Mong|Bloc, data=Dades)

	Friedman rank sum test

data:  Prod and Mong and Bloc
Friedman chi-squared = 18.571, df = 5, 
  p-value = 0.002309
\end{lstlisting}

\end{frame}


\begin{frame}
\vfill
\begin{center}
\gray{\LARGE\bf ANOVA de 2 vies}
\end{center}
\end{frame}


\begin{frame}
\frametitle{ANOVA de 2 vies}

Ens pot interessar comparar les mitjanes d'una variable sobre subpoblacions definides per més d'un factor a partir de mostres d'aquestes subpoblacions:  se'n diu un \emph{experiment factorial}\
\bigskip

Aquí considerarem només el  cas més senzill: el disseny d'\emph{ANOVA de 2 vies} (\emph{completament aleatori}):\medskip

\begin{itemize}
\item Empram els nivells (\red{tractaments}) de \red{dos factors} (\emph{2 vies})
 per classificar \medskip

\item Prenem mostres aleatòries independents \blue{de la mateixa mida} de cada combinació de nivells dels dos factors (\emph{completament aleatori})
\end{itemize}
\end{frame}



\begin{frame}
\frametitle{Exemple}

En un experiment per determinar l'atracció de moscards per colors i tipus de mel, s'han emprat 2 colors (vermell i verd) i 3 esquers (mel de taronger, mel de romaní i aigua), s'han oferit pots amb l'esquer tenyit del color a  grups  de moscards i s'ha mesurat el percentatge de moscards que s'han vist atrets pel pot. S'ha repetit 4 vegades per combinació (esquer, color) amb grups independents de moscards. Resultats:\vspace*{-2ex}

\begin{center}
\begin{tabular}{c|cc|cc|cc|}
\multicolumn{1}{c}{ }&\multicolumn{6}{c}{Esquer}\\\cline{2-7}
Color & \multicolumn{2}{|c|}{Taronger}& \multicolumn{2}{|c|}{Romaní} & \multicolumn{2}{|c|}{Aigua}\\\hline
Verd &65 &42 & 67 & 73 & 35 & 37  \\
& 53 &37 & 67  &70& 43 & 43\\\hline
 Vemell & 57 & 38  &60& 42& 35 & 33 \\
 &45 & 51 & 41& 68& 41 & 21 \\\hline
\end{tabular}
\end{center}

\blue{El color i el tipus de mel, afecten el percentatge mitjà de moscards atrets?}

\end{frame}


\begin{frame}
\frametitle{Exemple}


\red{Variable poblacional global}: \medskip

\begin{itemize}
\item $X$: Prenc un esbart de moscards i mir quin percentatge és atret per un pot contenint un esquer colorejat\medskip
\end{itemize}

\red{Subpoblacions}: Definides per les combinacions de \medskip

\begin{itemize}
\item Els dos colors (G: Verd, R: Vermell)\medskip

\item Els tres esquers (MT: Mel de taronger, MR: Mel de romaní, A: Aigua)
\end{itemize}
\end{frame}


\begin{frame}
\frametitle{Exemple}

\red{Variables d'interès}: \medskip

\begin{itemize}
\item $X_{G},X_{R}$: Prenc un esbart de moscards i mir quin percentatge és atret per un pot contenint un esquer colorejat de verd (G) o de vermell (R)\medskip

\item $X_{MT},X_{MR},X_A$: Prenc un esbart de moscards i mir quin percentatge és atret per un pot contenint mel de taronger (MT),  mel de romaní (MR) o aigua (A)\medskip

\item $X_{MT,G},X_{MR,G},X_{A,G},X_{MT,R},X_{MR,R},X_{A,R}$: Prenc un esbart de moscards i mir quin percentatge és atret per un pot contenint l'esquer corresponent tenyit del color corresponent
\end{itemize}
\end{frame}


\begin{frame}
\frametitle{Exemple}\vspace*{-1ex}

\begin{itemize}
\item \red{Hi ha diferència segons el color?}
$$
\left\{
\begin{array}{l}
H_0 : \mu_G=\mu_{R} \\
H_1 : \mu_G\neq \mu_{R}
\end{array}
\right.
$$\pause

\item \red{Hi ha diferència segons l'esquer?}
$$
\left\{
\begin{array}{l}
H_0 : \mu_{MT}=\mu_{MR}=\mu_A \\
H_1 : \text{ No és veritat que \ldots}
\end{array}
\right.
$$\pause

\item \red{Hi ha diferència segons la combinació d'esquer i color?}
$$
\left\{
\begin{array}{l}
H_0 : \mu_{E,C}=\mu_{E'C'}\text{ per a tots esquers $E,E'$ i colors $C,C'$} \\
H_1 : \text{ No és veritat que \ldots}
\end{array}
\right.
$$\pause

\item \red{Hi ha interacció entre colors i esquers?}
$$
\left\{
\begin{array}{l}
H_0 : \text{No hi ha interacció entre els esquers i els colors} \\
H_1 : \text{Hi ha interacció entre alguns esquers i alguns colors}
\end{array}
\right.
$$
\end{itemize}
\end{frame}


\begin{frame}
\frametitle{Situació general} 

Tenim una v.a. $X$ definida sobre una població\medskip

Classificam la població en subpoblacions segons dos factors, A i B.  El factor A té \red{$a$} nivells i el factor B, \red{$b$} nivells.
\medskip

\blue{Al nostre exemple}:
\begin{itemize}
\item \red{Factor A}: Esquer, $a=3$
\item \red{Factor B}: Color, $b=2$
\end{itemize}

\end{frame}


\begin{frame}
\frametitle{Situació general} 

\begin{itemize}
\item \red{$\mu$}: mitjana poblacional de $X$ global\medskip

\item \red{$X_{i\bullet}$}: Prenc un individu del  nivell $i$-èsim del factor A i hi mesur $X$,  $i=1,\ldots,a$; la seva mitjana  és \red{$\mu_{i\bullet}$}\medskip

\item \red{$X_{\bullet j}$}: Prenc un individu del  nivell $j$-èsim del factor B i hi mesur $X$, $j=1,\ldots,b$; la seva mitjana  és  \red{$\mu_{\bullet j}$}\medskip

\item  \red{$X_{ij}$}: Prenc un individu del  nivell $i$-èsim del factor A i el  nivell $j$-èsim del factor B i hi mesur $X$, de mitjana  \red{$\mu_{ij}$}
\end{itemize}

\end{frame}


\begin{frame}
\frametitle{Situació general} 


Prenem mostres de mida $n$ de cada $X_{ij}$
\begin{center}
\footnotesize
\begin{tabular}{ccccc}
\hline
&\multicolumn{4}{c}{Factor A}\\\hline
Factor B&$1$&$2$&$\cdots$&$a$\\\hline
$1$&$X_{111}$&$X_{211}$&$\cdots$&$X_{a11}$\\
&$\cdots$&$\cdots$&$\cdots$&$\cdots$\\
&$X_{11n}$&$X_{21n}$&$\cdots$&$X_{a1n}$\\\hline
$2$&$X_{121}$&$X_{221}$&$\cdots$&$X_{a21}$\\
&$\cdots$&$\cdots$&$\cdots$&$\cdots$\\
&$X_{12n}$&$X_{22n}$&$\cdots$&$X_{a2n}$\\\hline
$\vdots$&$\vdots$&$\vdots$&$\vdots$&$\vdots$\\\hline
$b$&$X_{1b1}$&$X_{2b1}$&$\cdots$&$X_{ab1}$\\
&$\cdots$&$\cdots$&$\cdots$&$\cdots$\\
&$X_{1bn}$&$X_{2bn}$&$\cdots$&$X_{abn}$\\\hline
\end{tabular}
\end{center}
\begin{itemize}
\item Les mostres de cada $X_{i\bullet}$ tenen mida $bn$
\item Les mostres de cada $X_{\bullet j}$ tenen mida $an$
\item  El nombre total d'observacions és $n\cdot a\cdot b$.
\end{itemize}

\end{frame}

\begin{frame}
\frametitle{Exemple} 

\begin{center}
\begin{tabular}{cccc}\hline
\multicolumn{1}{c}{ }&\multicolumn{3}{c}{Esquer}\\\hline
\multicolumn{1}{c}{Color} & \multicolumn{1}{c}{MT}&  \multicolumn{1}{c}{MR} & \multicolumn{1}{c}{A}  \\\hline
G &65 &  67 &  35   \\
&42& 73& 37 \\
& 53  & 67 &  43 \\
&37&  70&   43 \\\hline
R & 57  & 60& 35  \\
 & 38& 42& 33\\
 &45 & 41& 41 \\
 & 51& 68&  21 \\\hline
 \end{tabular}
\end{center}

\end{frame}


\begin{frame}
\frametitle{Situació general} 

\begin{itemize}

\item $\red{\overline{X}_{i\bullet}}$: mitjana mostral del nivell $i$-èsim d'A 
$$
\overline{X}_{i}=\dfrac{\sum_{j=1}^b X_{ij}}{b\cdot n}
$$

\item $\red{\overline{X}_{\bullet j}}$: mitjana mostral del nivell $j$-èsim  de  B 
$$
\overline{X}_{\bullet j}=\dfrac{\sum_{i=1}^k X_{ij}}{a\cdot n}
$$

\item $\red{\overline{X}_{i j}}$: mitjana mostral de la combinació del nivell $i$-èsim del factor A i el nivell $j$-èsim  del factor B 
$$
\overline{X}_{i j}=\dfrac{\sum_{i=1}^k X_{ij}}{n}
$$


\item $\red{\overline{X}}$: mitjana mostral global 
$$
\overline{X}=\dfrac{\sum_{i=1}^k\sum_{j=1}^b X_{ij}}{a\cdot b\cdot n}
$$
\end{itemize}
\end{frame}

\begin{frame}
\frametitle{Exemple} 

\begin{center}
\begin{tabular}{c|cc|cc|cc|c}
\multicolumn{1}{c}{ }&\multicolumn{6}{c}{Esquer}\\\cline{2-7}
Color &\multicolumn{2}{|c|}{Taronger}& \multicolumn{2}{|c|}{Romaní} & \multicolumn{2}{|c|}{Aigua} & \\\hline
Verd &65 & & 67 & & 35 &  \\
&42& &73& &37& \\
& 53  && 67 & & 43& \\
&37&  \hspace*{-0.5cm} $\overline{X}_{11}$& 70&  \hspace*{-0.5cm} $\overline{X}_{21}$& 43&  \hspace*{-0.7cm} $\overline{X}_{31}$ & $\overline{X}_{\bullet 1}$\\\hline
Vemell & 57  & &60&& 35 & \\
 & 38&& 42&& 33&\\
 &45 && 41&& 41 &\\
 & 51&  \hspace*{-0.5cm} $\overline{X}_{12}$& 68&  \hspace*{-0.5cm} $\overline{X}_{22}$& 21&  \hspace*{-0.7cm} $\overline{X}_{32}$& $\overline{X}_{\bullet2}$ \\\hline
\vphantom{$\Big($} & $\overline{X}_{1\bullet}$ & & $\overline{X}_{2\bullet}$ & & $\overline{X}_{3\bullet}$ & \\
 \multicolumn{1}{c}{}& \multicolumn{6}{c}{$\underbrace{\hphantom{\overline{X}_{11}+\overline{X}_{11}+\overline{X}_{11}+\overline{X}_{11}+\overline{X}_{11}}}_{\displaystyle \overline{X}}$} & \multicolumn{1}{c}{}
 \end{tabular}
\end{center}

\end{frame}



\begin{frame}
\frametitle{Exemple} 

Organitzarem les dades del nostre exemple en un dataframe  amb 3 variables:
\medskip

\begin{itemize}
\item \blue{{\tt Percent}}, quantitativa, el percentatge de moscards atrets\medskip

\item \blue{{\tt Esquer}}, un factor que  contendrà el valor del nivell  del  factor
A (esquer) per a cada grup de moscards: MT, MR, A
\medskip

\item \blue{{\tt Color}}, un factor que   contendrà el valor del nivell del  factor
B (color) per a cada grup de moscards: G, R
\end{itemize}
\end{frame}



\begin{frame}
\frametitle{Exemple} 

\begin{center}
\begin{tabular}{cccc}\hline
\multicolumn{1}{c}{ }&\multicolumn{3}{c}{Esquer}\\\hline
\multicolumn{1}{c}{Color} & \multicolumn{1}{c}{MT}&  \multicolumn{1}{c}{MR} & \multicolumn{1}{c}{A}  \\\hline
G &65 &  67 &  35   \\
&42& 73& 37 \\
& 53  & 67 &  43 \\
&37&  70&   43 \\\hline
R & 57  & 60& 35  \\
 & 38& 42& 33\\
 &45 & 41& 41 \\
 & 51& 68&  21 \\\hline
 \end{tabular}
\end{center}

\end{frame}





\begin{frame}[fragile]
\frametitle{Exemple} 

\begin{lstlisting}
> Percent=c(65,67,35,42,73,37,
   53,67,43,37,70,43,57,60,35,
   38,42,33,45,41,41,51,68,21)
> Esquer=factor(rep(c("MT","MR","A"),times=8),
               levels=c("MT","MR","A"))
> Esquer
 [1] MT MR A  MT MR A  MT MR A  MT MR
[12] A  MT MR A  MT MR A  MT MR A  MT
[23] MR A 
Levels: MT MR A
> Color=factor(rep(c("G","R"),each=12),
              levels=c("G","R"))
> Color
 [1] G G G G G G G G G G G G R R R R
[17] R R R R R R R R
Levels: G R
\end{lstlisting}


\end{frame}



\begin{frame}[fragile]
\frametitle{Exemple} 


\begin{lstlisting}
> str(Moscards)
'data.frame':	24 obs. of  3 variables:
 $ Percent: num  65 67 35 42 73 37 53 67 43 37 ...
 $ Esquer : Factor w/ 3 levels "MT","MR","A": 1 2 3 1 2 3 1 2 3 1 ...
 $ Color  : Factor w/ 2 levels "G","R": 1 1 1 1 1 1 1 1 1 1 ...
\end{lstlisting}
\end{frame}



\begin{frame}[fragile]
\frametitle{Exemple} 

\begin{lstlisting}
> Xb.i.b=aggregate(Percent~Esquer,data=Moscards,mean)
> Xb.i.b
  Esquer Percent
  Esquer Percent
1     MT    48.5
2     MR    61.0
3      A    36.0
> Xb.b.j=aggregate(Percent~Color,data=Moscards,mean)
> Xb.b.j
  Color  Percent
1     G 52.66667
2     R 44.33333
> Xb=mean(Percent); Xb
[1] 48.5
\end{lstlisting}\vspace*{-1ex}

$$
\begin{array}{cccccccc}
\overline{X}_{MT} & \overline{X}_{MR} & \overline{X}_A & \hspace*{0.5cm} & \overline{X}_G & \overline{X}_R& \hspace*{0.5cm} & \overline{X}\\ \hline
48.5 & 61 & 36 & & 52.67 & 44.33 & & 48.5
\end{array}
$$
\end{frame}



\begin{frame}[fragile]
\frametitle{Exemple} 

\begin{lstlisting}
> Xb.i.j=aggregate(Percent~Esquer+Color,
   data=Moscards,mean)
> Xb.i.j
  Esquer Color Percent
1     MT     G   49.25
2     MR     G   69.25
3      A     G   39.50
4     MT     R   47.75
5     MR     R   52.75
6      A     R   32.50
\end{lstlisting}

\begin{center}
\begin{tabular}{c|ccc}
$\overline{X}_{ij}$ & MT & MR & A\\ \hline
G & 49.25 & 69.25 & 39.50\\
R &  47.75 & 52.75 &  32.50
\end{tabular}
\end{center}

\end{frame}





\begin{frame}
\frametitle{Condicions necessàries} 

Per poder fer una ANOVA de 2 vies, cal que:\medskip

\begin{itemize}
\item Les observacions per a cada combinació de nivells $(i,j)$ constitueixin
m.a.s. independents de les variables $X_{ij}$, totes de la mateixa  mida $n$
\medskip 

\item Les variables $X_{ij}$ siguin totes normals 
\medskip 

\item \red{Homocedasticitat}: Les variables $X_{ij}$ tenguin totes la mateixa variància,
$\sigma^2$
\end{itemize}

\end{frame}

\begin{frame}
\frametitle{Model}\vspace*{-3ex}


$$
X_{ij} = \mu + (\mu_{i\bullet}-\mu) +(\mu_{\bullet j}-\mu) + (\alpha\beta)_{ij}+E_{ij}
$$
on
\begin{itemize}
\item \red{$\mu_{i\bullet}-\mu$}: \red{Efecte del factor A}\medskip

\item \red{$\mu_{\bullet j}-\mu$}: \red{Efecte del factor B}\medskip

\item $\red{(\alpha\beta)_{ij}}$ ($=\mu_{ij}-\mu_{i\bullet}-\mu_{\bullet j}+\mu$): \red{Efecte de la interacció entre el  nivell $i$-èsim del factor A i el  nivell  $j$-èsim del factor B}
\medskip

\item $\red{E_{ij}}$ ($=X_{ij}-\mu_{ij}$): \red{Residu},  \red{Error aleatori}
\end{itemize}
\end{frame}


\begin{frame}
\frametitle{Identitats de les sumes de quadrats}\vspace*{-2ex} 

\begin{teorema}
$SS_{Total} = SS_{Tr}+SS_E$\\[1ex]
$SS_{Tr} = SS_A+SS_B+SS_{AB}$
\end{teorema}

\begin{itemize}
\item $\red{SS_{Total}} =
\sum\limits_{i=1}^a\sum\limits_{j=1}^b\sum\limits_{k=1}^n
(X_{ijk}-\overline{X}_{\bullet\bullet})^2$: \red{Suma de Total de Quadrats}, representa la \blue{variabilitat global de la mostra} 


\item $\red{SS_{Tr}}=n\sum\limits_{i=1}^a\sum\limits_{j=1}^b
(\overline{X}_{ij}-\overline{X}_{\bullet\bullet})^2$: \red{Suma de Quadrats dels Tractaments}, representa la \blue{variabilitat 
de les mitjanes de les combinacions de tractaments d'A i B}




\item $\red{SS_E} =
\sum\limits_{i=1}^a\sum\limits_{j=1}^b\sum\limits_{k=1}^n
(X_{ijk}-\overline{X}_{ij})^2$: \red{Suma de Quadrats dels  Errors}, representa la \blue{variabilitat  deguda a factors aleatoris}
\end{itemize}

\end{frame}



\begin{frame}
\frametitle{Identitats de les sumes de quadrats}\vspace*{-2ex} 

\begin{teorema}
$SS_{Total} = SS_{Tr}+SS_E$\\[1ex]
$SS_{Tr} = SS_A+SS_B+SS_{AB}$
\end{teorema}

\begin{itemize}

\item  $\red{SS_A} =b n\sum\limits_{i=1}^a
(\overline{X}_{i\bullet}-\overline{X}_{\bullet\bullet})^2$: \red{Suma de Quadrats del factor A}, representa la \blue{variabilitat 
de les mitjanes dels nivells d'A}


\item  $\red{SS_B}\! =\!a n\!\sum\limits_{j=1}^b (\overline{X}_{\bullet
j}-\overline{X}_{\bullet\bullet})^2$: \red{Suma de Quadrats del factor~B}, representa la \blue{variabilitat 
de les mitjanes dels nivells de~B}

\item $\red{SS_{AB}}=n \sum\limits_{i=1}^a\sum\limits_{j=1}^b
(\overline{X}_{ij}-\overline{X}_{i\bullet}-\overline{X}_{\bullet
j}+\overline{X}_{\bullet\bullet})^2$: \red{Suma de Quadrats de la Interacció}, representa la \blue{variabilitat 
deguda a la interacció dels nivells d'A i B}


\end{itemize}

\end{frame}


\begin{frame}
\frametitle{Identitats de les sumes de quadrats}\vspace*{-2ex} 

$$
\begin{array}{l}
\blue{SS_{Total} = SS_{Tr}+SS_E}\\[1ex]
\blue{SS_{Tr} = SS_A+SS_B+SS_{AB}}
\end{array}
$$
Per tant
$$
\red{SS_{Total} = SS_A+SS_B+SS_{AB}+SS_E}
 $$
\end{frame}





\begin{frame}[fragile]
\frametitle{Exemple}\vspace*{-2ex} 

\begin{lstlisting}
> n=4; a=3; b=2
> SS.Tot=sum((Percent-Xb)^2); SS.Tot
[1] 4636
> SS.A=n*b*sum((Xb.i.b[,2]-Xb)^2); SS.A
[1] 2500
> SS.B=n*a*sum((Xb.b.j[,2]-Xb)^2); SS.B
[1] 416.6667
> SS.Tr=n*sum((Xb.i.j[,3]-Xb)^2); SS.Tr
[1] 3147
> SS.AB=n*sum((Xb.i.j[,3]-Xb.i.b[,2]
   -rep(Xb.b.j[,2],each=a)+Xb)^2)
> SS.AB
[1] 230.3333
> SS.E=sum((Percent-c(rep(Xb.i.j[1:a,3],n),
   rep(Xb.i.j[(a+1):(2*a),3],n)))^2)
> SS.E
[1] 1489
\end{lstlisting}

\end{frame}






\begin{frame}[fragile]
\frametitle{Exemple}

\begin{center}
\begin{tabular}{cccccc}
$SS_{Total}$ & $SS_A$ & $SS_B$ &   $SS_{AB}$ &$SS_{Tr}$ &  $SS_E$\\ \hline
4636 & 2500  & 416.67  &   230.33  & 3147  &  1489
\end{tabular}
\end{center}

\begin{lstlisting}
> SS.A+SS.B+SS.AB
[1] 3147
> SS.Tr+SS.E
[1] 4636
\end{lstlisting}


\end{frame}


\begin{frame}
\frametitle{Quadrats mitjans}

\begin{itemize}
\item  \red{Quadrat mitjà del factor A}: 
$\red{MS_A} =\dfrac{SS_A}{a-1}$\bigskip

\item  \red{Quadrat mitjà del factor B}: 
$\red{MS_B} =\dfrac{SS_B}{b-1}$\bigskip


\item  \red{Quadrat mitjà dels tractaments}: 
$\red{MS_{Tr}}=\dfrac{SS_{Tr}}{ab-1}$\bigskip


\item  \red{Quadrat mitjà de la interacció}: 
$\red{MS_{AB}}=\dfrac{SS_{AB}}{(a-1)(b-1)}$\bigskip


\item  \red{Quadrat mitjà dels errors}: 
$\red{MS_E}=\dfrac{SS_E}{ab (n-1)}$

\end{itemize}
\end{frame}



\begin{frame}[fragile]
\frametitle{Exemple}

$n=4$, $a=3$, $b=2$
\medskip

\begin{center}
\begin{tabular}{cccccc}
$SS_{Total}$ & $SS_A$ & $SS_B$ &   $SS_{AB}$ &$SS_{Tr}$ &  $SS_E$\\ \hline
4636 & 2500  & 416.67  &   230.33  & 3147  &  1489
\end{tabular}
\bigskip

\begin{tabular}{ccccc}
$MS_A$ & $MS_B$ & $MS_{AB}$ & $MS_{Tr}$ & $MS_E$\\ \hline
1250 &  416.67 &  115.17 & 629.4 &   82.72
\end{tabular}
\end{center}
\begin{lstlisting}
> MS.B=SS.B/(b-1)
> MS.AB=SS.AB/((a-1)*(b-1))
> MS.Tr=SS.Tr/(a*b-1)
> MS.E=SS.E/(a*b*(n-1))
> round(c(MS.A,MS.B,MS.AB,MS.Tr,MS.E),2)
[1] 1250.00  416.67  115.17  629.40  82.72
\end{lstlisting}

\end{frame}



\begin{frame}
\frametitle{Contrast de mitjanes del factor A}

 Contrastam si  hi ha diferències entre les mitjanes dels nivells del factor A:
$$
\left\{
\begin{array}{l}
H_0 : \mu_{1\bullet}=\mu_{2\bullet}=\cdots
=\mu_{a\bullet} \\
H_1 :  \mbox{Hi ha } i,i'\mbox{ tals que }  \mu_{i\bullet}
\not = \mu_{i'\bullet}
\end{array}
\right.
$$
\medskip

L'estadístic de contrast és
$$
\red{F_A=\frac{MS_A}{MS_E}},
$$
Si $H_0$ és certa, té
distribució $F$ de Fisher amb $a-1$ i $ab(n-1)$ graus de llibertat i valor proper a 1

\end{frame}


\begin{frame}
\frametitle{Contrast de mitjanes del factor B}
 Contrastam si  hi ha diferències entre les mitjanes dels nivells del factor  B:
$$
\left\{
\begin{array}{l}
H_0 : \mu_{\bullet 1}=\mu_{\bullet 2}=\cdots =\mu_{ \bullet
b} \\
H_1 :  \mbox{Hi ha } j,j'\mbox{ tals que }  \mu_{\bullet j}
\not = \mu_{\bullet j'}
\end{array}
\right.
$$
\medskip

L'estadístic de contrast és
$$
\red{F_B=\frac{MS_B}{MS_E}},
$$
Si $H_0$ és certa, té
distribució $F$ de Fisher amb $b-1$ i $ab(n-1)$ graus de llibertat i valor proper a 1

\end{frame}



\begin{frame}
\frametitle{Contrast dels tractaments}

 Contrastam si  hi ha diferències entre les mitjanes de  les parelles (nivell de  A, nivell de  B):
$$
\left\{
\begin{array}{l}
H_0 : \mbox{Per a tots }i,j,i',j',\ \mu_{ij}=\mu_{i'j'} \\
H_1 :  \mbox{Hi ha } i,j,i',j'\mbox{ tals que } \mu_{ij}\neq \mu_{i'j'}
\end{array}
\right.
$$
\medskip

L'estadístic de contrast és
$$
\red{F_{Tr}=\frac{MS_{Tr}}{MS_E}},
$$
Si $H_0$ és certa, té
distribució $F$ de Fisher amb $ab-1$ i $ab(n-1)$ graus de llibertat i valor proper a 1

\end{frame}

\begin{frame}
\frametitle{Contrast de no interacció}

 Contrastam si  hi ha interacció entre
els factors A i B:
$$
\left\{
\begin{array}{l}
H_0 :  \mbox{Per a tots } i,j, (\alpha\beta)_{ij} =0 \\
H_1 :  \mbox{Hi ha } i,j\mbox{ tals que } (\alpha\beta)_{ij}
\not = 0
\end{array}
\right.
$$
\medskip

L'estadístic de contrast és
$$
\red{F_{AB} = \frac{MS_{AB}}{MS_E}},
$$
Si $H_0$ és certa, té
distribució $F$ de Fisher amb $(a-1)(b-1)$ i $ab(n-1)$ graus de
llibertat  i valor proper a 1

\end{frame}


\begin{frame}
\frametitle{Contrastos}

En els quatre casos, el p-valor és
$$
P(F_{x,y}\geq \mbox{valor de l'estadístic})
$$
on $F_{x,y}$ representa la distribució $F$ de Fisher amb els graus de llibertat que pertoquin.

\end{frame}


\begin{frame}
\frametitle{Taula  ANOVA}

Els contrastos anteriors es resumeixen en la taula  ANOVA:

{\small \begin{center}
\begin{tabular}{cccccc}
\hline
 Font de& Graus de& Suma de& Quadrats& $F$ & p- \\
variació&llibertat&quadrats&mitjans& &\\\hline
Tract.&$ab-1$&$SS_{Tr}$&$MS_{Tr}$& $F_{Tr}$ & p-valor \\[1ex]
A&$a-1$&$SS_A$&$MS_{A}$& $F_{A}$&  p-valor\\[1ex]
B&$b-1$&$SS_B$&$MS_{B}$& $F_{B}$&  p-valor\\[1ex]
$AB$& {$(a-1)(b-1)$} &$SS_{AB}$&$MS_{AB}$& $F_{AB}$&  p-valor\\[1ex]
Error&$ab(n-1)$&$SS_E$&$MS_{E}$&& \\\hline
\end{tabular}
\end{center}
}
\end{frame}


\begin{frame}
\frametitle{Exemple}

$n=4$, $a=3$, $b=2$
\medskip

\begin{center}
\begin{tabular}{ccccc}
$MS_A$ & $MS_B$ & $MS_{AB}$ & $MS_{Tr}$ & $MS_E$\\ \hline
1250 &  416.67 &  115.17 & 629.4 &   82.72
\end{tabular}
\end{center}

$\dfrac{MS_{A}}{MS_E}=\dfrac{1250}{82.72}=15.11$\medskip 

p-valor$=P(F_{2,18}\geq 15.11)=\texttt{1-pf(15.11,2,18)}=10^{-4}$
\bigskip

Hem trobat evidència estadística que els percentatges mitjans d'esbarts moscards que són atrets pels diferents tipus d'esquer no són tots iguals (ANOVA de 2 vies, p-valor $10^{-4}$)

\end{frame}


\begin{frame}
\frametitle{Exemple}

$n=4$, $a=3$, $b=2$
\medskip

\begin{center}
\begin{tabular}{ccccc}
$MS_A$ & $MS_B$ & $MS_{AB}$ & $MS_{Tr}$ & $MS_E$\\ \hline
1250 &  416.67 &  115.17 & 629.4 &   82.72
\end{tabular}
\end{center}

$\dfrac{MS_{B}}{MS_E}=\dfrac{416.67}{82.72}=5.04$\medskip 

p-valor$=P(F_{1,18}\geq 5.04)=\texttt{1-pf(5.04,1,18)}=0.038$
\bigskip

Hem trobat evidència estadística que els percentatges mitjans d'esbarts moscards que són atrets pels  colors verd i vemell són diferents (ANOVA de 2 vies, p-valor 0.038)

\end{frame}

\begin{frame}\frametitle{Exemple}

$n=4$, $a=3$, $b=2$
\medskip

\begin{center}
\begin{tabular}{ccccc}
$MS_A$ & $MS_B$ & $MS_{AB}$ & $MS_{Tr}$ & $MS_E$\\ \hline
1250 &  416.67 &  115.17 & 629.4 &   82.72
\end{tabular}
\end{center}

$\dfrac{MS_{Tr}}{MS_E}=\dfrac{629.4}{82.72}=7.61$\medskip 

p-valor$=P(F_{5,18}\geq 7.61)=\texttt{1-pf(7.61,5,18)}=5\cdot 10^{-4}$
\bigskip

Hem trobat evidència estadística que els percentatges mitjans d'esbarts moscards que són atrets per les diferents combinacions d'esquer i color no són tots iguals (ANOVA de 2 vies, p-valor $5\cdot 10^{-4}$)

\end{frame}

\begin{frame}\frametitle{Exemple}

$n=4$, $a=3$, $b=2$
\medskip

\begin{center}
\begin{tabular}{ccccc}
$MS_A$ & $MS_B$ & $MS_{AB}$ & $MS_{Tr}$ & $MS_E$\\ \hline
1250 &  416.67 &  115.17 & 629.4 &   82.72
\end{tabular}
\end{center}

$\dfrac{MS_{AB}}{MS_E}=\dfrac{115.17}{82.72}=1.39$\medskip 

p-valor$=P(F_{2,18}\geq 1.39)=\texttt{1-pf(1.39,2,18)}=0.274$
\bigskip

No hem trobat evidència estadística d'interacció entre els tipus d'esquer i els colors (ANOVA de 2 vies, p-valor 0.274)

\end{frame}



\begin{frame}
\frametitle{Exemple}

\begin{center}\small
\begin{tabular}{cccccc}
\hline
 Font de& Graus de& Suma de& Quadrats& $F$ & p- \\
variació&llibertat&quadrats&mitjans& &\\\hline
Combs.&$5$& 3147&629.4& 7.61 & $5\cdot 10^{-4}$ \\[1ex]
Esquer &2& 2500 &1250& 15.11&  $10^{-4}$ \\[1ex]
Color &1&416.67&416.67& 5.04&  0.038\\[1ex]
Interacció & 2 &230.33&115.17& 1.39 &  0.274\\[1ex]
Error&30&1489&82.72 && \\\hline
\end{tabular}
\end{center}

\end{frame}


\begin{frame}[fragile]
\frametitle{Amb R}

 \begin{lstlisting}
> summary(aov(Percent~Esquer*Color, 
  data=Moscards))
             Df Sum Sq Mean Sq F value   Pr(>F)    
Esquer        2 2500.0  1250.0  15.111 0.000141 ***
Color         1  416.7   416.7   5.037 0.037622 *  
Esquer:Color  2  230.3   115.2   1.392 0.274038    
Residuals    18 1489.0    82.7                     
\end{lstlisting}\pause

\begin{lstlisting}
> summary(aov(Percent~Esquer:Color, 
  data=Moscards))
             Df Sum Sq Mean Sq F value   Pr(>F)    
Esquer:Color  5   3147   629.4   7.609 0.000538 ***
Residuals    18   1489    82.7   
\end{lstlisting}

\end{frame}



\begin{frame}[fragile]
\frametitle{Podíem emprar una ANOVA?}\vspace*{-2ex}

\begin{enumerate}
\item La variable d'interès sobre cada combinació de nivells és normal?
\end{enumerate}

\begin{lstlisting}
> es.normal=function(x){shapiro.test(x)$p.value}
> aggregate(Percent~Esquer+Color, data=Moscards, es.normal)
  Esquer Color   Percent
1     MT     G 0.7506899
2     MR     G 0.2724532
3      A     G 0.1612393
4     MT     R 0.9772990
5     MR     R 0.2674892
6      A     R 0.6499652
\end{lstlisting}

\end{frame}



\begin{frame}[fragile]
\frametitle{Podíem emprar una ANOVA?}\vspace*{-2ex}

\begin{enumerate}\setcounter{enumi}{1}
\item I totes amb la mateixa variància?
\end{enumerate}

\begin{lstlisting}
> bartlett.test(Percent~interaction(Esquer,Color),data=Moscards)

	Bartlett test of homogeneity of variances

data:  Percent by interaction(Esquer, Color)
Bartlett's K-squared = 7.6201, df = 5, 
p-value = 0.1785
\end{lstlisting}


\end{frame}



\begin{frame}[fragile]
\frametitle{Alerta} 

Les condicions sobre les combinacions de nivells no impliquen les condicions sobre cada nivell per separat\medskip


\begin{lstlisting}
> set.seed(100)
> mostra.X.1=rnorm(10,0,1)
> mostra.X.2=rnorm(10,5,1)
> mostra.Y.1=rnorm(10,0,1)
> mostra.Y.2=rnorm(10,10,1)
> x=c(mostra.X.1,mostra.X.2)
> y=c(mostra.Y.1,mostra.Y.2)
> shapiro.test(x)$p.value
[1] 0.003534705
> shapiro.test(y)$p.value
[1] 0.0001874323
> fligner.test(list(x,y))$p.value
[1] 1.245335e-06
\end{lstlisting}

\end{frame}





\begin{frame}[fragile]
\frametitle{Comparacions posteriors per parelles} 

Les comparacions posteriors per parelles de les mitjanes de les combinacions de nivells les podeu fer amb \texttt{pairwise.t.test}

\begin{lstlisting}
> pairwise.t.test(Moscards$Percent,   
  Moscards$Esquer:Moscards$Color, paired=FALSE)

...
     MT:G    MT:R    MR:G    MR:R    A:G    
MT:R 1.00000 -       -       -       -      
MR:G 0.06665 0.04706 -       -       -      
MR:R 1.00000 1.00000 0.17930 -       -      
A:G  0.88126 1.00000 0.00294 0.37892 -      
A:R  0.17930 0.23273 0.00031 0.06665 1.00000

P value adjustment method: holm 
\end{lstlisting}
\pause
 
 \blue{La resta, ho deixarem córrer}

\end{frame}


\end{document}



\begin{frame}[fragile]
\red{Si no hi ha interacció}, també podeu fer les comparacions posteriors per parelles de les mitjanes dels tractaments de cada factor amb \texttt{pairwise.t.test}
\begin{lstlisting}
> pairwise.t.test(Moscards$Percent, Moscards$Esquer,
  paired=FALSE, p.adjust.method="none")

	Pairwise comparisons using t tests with pooled SD 

data:  Moscards$Percent and Moscards$Esquer 

   MT    MR     
MR 0.022 -      
A  0.022 6.6e-05

P value adjustment method: none 
> t.test(Percent~Color, data=Moscards,
  paired=FALSE, var.equal=TRUE)$p.value
[1] 0.1546603
\end{lstlisting}
\end{frame}




\end{document}

\subsection{No paramètric}

\begin{frame} 
\frametitle{Contrast no paramètric}\vspace*{-2ex}

Si en un experiment  de 2 vies no podem realitzar una ANOVA perquè no se satisfan les condicions necessàries, \red{cal emprar  un test no paramètric}\medskip

No hi ha cap opció  senzilla si hi ha interacció \medskip

Si sospitau que no hi ha interacció, us recomanam:\smallskip
\begin{itemize}
\item Per  al contrast de Tractaments, emprar el test de Kruskal-Wallis per a la combinació de factors (definida a la fórmula que hi entrau amb \texttt{interaction})\medskip

\item Per al  contrast de mitjanes de cada factor,  calcular les mitjanes mostrals per combinacions de nivells i llavors emprar el test de Friedman com si aquestes mitjanes provinguessin de blocs (però tendrà poca potència)
\end{itemize}



\end{frame}


\begin{frame}[fragile]
\frametitle{Exemple 4}

Contrast de tractaments:\medskip

\begin{lstlisting}
> kruskal.test(GSI~interaction(llum,temp),data=peixos)

	Kruskal-Wallis rank sum test

data:  GSI by interaction(llum, temp)
Kruskal-Wallis chi-squared = 15.594, df = 3, p-value = 0.001373
\end{lstlisting}

\end{frame}


\begin{frame}[fragile]
\frametitle{Exemple 4}

Ja havíem calculat les mitjanes mostrals per combinacions de nivells\medskip

\begin{lstlisting}
> Xb.i.j.bullet=aggregate(GSI~llum+temp,
   data=peixos, FUN=mean)
\end{lstlisting}
   
Contrast de les mitjanes del factor \texttt{llum}:\medskip

\begin{lstlisting}
> friedman.test(GSI~llum|temp, data=Xb.i.j.bullet)

	Friedman rank sum test

data:  GSI and llum and temp
Friedman chi-squared = 2, df = 1, p-value = 0.1573
\end{lstlisting}\medskip


\end{frame}



\end{document}
